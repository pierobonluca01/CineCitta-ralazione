\section{Analisi}
\subsection{Target di utenza}

Il sito web è stato sviluppato per offrire servizi e informazioni relative al mondo del cinema. Gli utenti possono consultare i film in programmazione e gli orari delle relative proiezioni, oltre ad avere la possibilità di effettuare prenotazioni per le proiezioni desiderate.

Il sito mira a soddisfare le esigenze degli appassionati di cinema di tutte le età ed interessi. Il cinema offre anche sconti e promozioni mirate a diversi target di utenza, al fine di rendere l'esperienza cinematografica accessibile a tutti.

Il target di utenza è quindi estremamente diversificato e comprende sia gli amanti del cinema che desiderano esplorare una vasta gamma di generi e film, sia gruppi di amici o famiglie in cerca di opzioni di intrattenimento più generali. Vogliamo garantire che ogni utente possa trovare informazioni dettagliate sulle proiezioni, inclusi gli orari e le informazioni del film, come la sua sinossi, in modo da prendere decisioni informate sulla loro esperienza cinematografica.

Si può suddividere la ricerca di queste informazioni nel sito attraverso la metafora della pesca:
\begin{itemize}
    \item \textbf{Tiro perfetto:} L'utente può raggiungere l'informazione che cerca al primo colpo come ad esempio le proiezioni per il giorno corrente o l'elenco degli orari per data o per film specifico.
    \item \textbf{Trappola per aragoste:} L'interfaccia intuitiva e facile da navigare consente agli utenti di ritrovare facilmente le informazioni di cui hanno bisogno nelle prossime iterazioni di ricerca. 
    \item \textbf{Pesca con la rete:} La struttura del sito è semplice ed essenziale per garantire un'esperienza utente ottimale, questo consente agli utenti di esplorare il sito senza perdere contenuti o avere un sovraccarico cognitivo.
    \item \textbf{Boa di segnalazione:} Le pagine visibili all'utente utilizzano il passaggio di parametri tramite metodo GET come ad esempio la scheda di un film e le relative proiezioni. Questo consente all'utente di aggiungere le pagine ai preferiti per poterle visitare in un secondo momento. Inoltre se l'utente è registrato può ritrovare l'elenco delle prenotazioni effettuate in qualsiasi momento attraverso la propria dashboard.
\end{itemize}

\subsection{Servizi}
\begin{itemize}
    \item \textbf{Film in evidenza:} Appena raggiunge il sito, l'utente può visualizzare tre film tra quelli in proiezione e, se interessato, visualizzarne direttamente la lista degli orari.
    \item \textbf{Oggi in sala:} Sempre in homepage, l'utente visualizza immediatamente l'elenco delle proiezioni per la giornata corrente.
    \item \textbf{Film in sala:} Pagina dedicata all'elenco di tutti i film disponibili.
    \item \textbf{Proiezioni:} Elenco degli orari di proiezione suddivisi per data.
    \item \textbf{Scheda di un film:} Pagina di informazioni riguardanti il film selezionato (da qualsiasi pagina) con elenco degli orari di proiezione per ogni data.
    \item \textbf{Contatti:} Pagina che raccoglie le principali informazioni della struttura. Su questa pagina, gli utenti possono trovare dettagli come l'indirizzo, i contatti e gli orari di apertura.
    È anche possibile inviare un messaggio utilizzando il form di contatto.
    \item \textbf{Prenotazioni:} Possibilità, da utente registrato, di riservarsi i posti per una proiezione scelta. All'utente verrà fornito un codice univoco da presentare in cassa per ultimare il pagamento.
\end{itemize}
